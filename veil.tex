\section{Veil}

\subsection{Motivation}

Web browsers are the client-side execution platform
for a variety of online services. Many of these
services handle sensitive personal data like emails
and financial transactions. Since a user's machine
may be shared with other people, she may wish to
establish a \emph{private session} with a web site,
such that the session leaves no persistent
client-side state that can later be examined by a
third party. Even if a site does not handle
personally identifiable information, users may
not want to leave evidence that a site was even
visited. Thus, all popular browsers implement a
private browsing mode which tries to remove
artifacts like entries in the browser's ``recently
visited'' URL list.

Unfortunately, implementations of private browsing
mode still allow sensitive information to leak into
persistent storage~\cite{aggarwal10,dt2016,magnetForensicsChrome,ohana13}.
Browsers use the file system or a SQLite database
to temporarily store information associated with private
sessions; this data is often incompletely deleted
and zeroed-out when a private session terminates,
allowing attackers to extract images and URLs from
the session. During a private session, web page
state can also be reflected from RAM into swap
files and hibernation files; this state is in cleartext,
and therefore easily analyzed by curious individuals
who control a user's machine after her private
browsing session has ended. Simple greps for
keywords are often sufficient to reveal sensitive
data~\cite{aggarwal10,dt2016}.

Web browsers are complicated platforms that are
continually adding new features (and thus new ways
for private information to leak). As a result, it is
difficult to implement even seemingly straightforward
approaches for strengthening a browser's implementation
of incognito modes. For example, to prevent secrets in
RAM from paging to disk, the browser could use OS
interfaces like \texttt{mlock()} to pin memory pages.
However, pinning may interfere in subtle ways with other
memory-related functionality like garbage collecting or
tab discarding~\cite{tabDiscarding}.
% which tries to minimize RAM usage by unloading the
% content in dormant tabs.
Furthermore, the browser would have to use \texttt{mlock()}
indiscriminately, on \textit{all} of the RAM state
belonging to a private session, because the browser would
have no way to determine which state in the session is
actually sensitive, and which state can be safely
exposed to the swap device.

In this thesis, we introduce Veil, a system
that allows web developers to implement
private browsing semantics for their own pages.
For example, the developers of a whisteblowing
site can use Veil to reduce the likelihood
that employers can find evidence of visits to
the site on workplace machines.
Veil's privacy-preserving mechanisms are
enforced \emph{without assistance from the
browser}---even if users visit pages using
a browser's built-in privacy mode, Veil
provides stronger assurances that
can only emerge from an intentional
composition of HTML, CSS, and JavaScript.
Veil leverages five techniques to protect
privacy: URL blinding, content mutation,
heap walking, DOM hiding, and state encryption.
  \begin{itemize}
    \item Developers pass their HTML and CSS files through
          Veil's compiler. The compiler locates cleartext
          URLs in the content, and transforms those raw
          URLs into \emph{blinded references} that are
          derived from a user's secret key and are
          cryptographically unlinkable to the original
          URLs. The compiler also injects a runtime
          library into each page; this library interposes
          on dynamic content fetches (e.g., via
          \texttt{XMLHttpRequests}), and forces those
          requests to also use blinded references.
    \item The compiler uploads the objects in a web page to
          Veil's \emph{blinding servers}. A user's browser
          downloads content from those blinding servers, and
          the servers collaborate with a page's JavaScript
          code to implement the blinded URL protocol. To
          protect the client-side memory artifacts belonging
          to a page, the blinding servers also \emph{dynamically
          mutate} the HTML, CSS, and JavaScript in a page.
          Whenever a user fetches a page, the blinding servers
          create syntactically different (but semantically
          equivalent) versions of the page's content. This
          ensures that two different users of a page will
          each receive unique client-side representations
          of that page.
    \item Ideally, sensitive memory artifacts would never
          swap out in the first place. Veil allows developers
          to mark JavaScript state and renderer state as
          sensitive. Veil's compiler injects \emph{heap
          walking code} to keep that state from swapping out.
          The code uses JavaScript reflection
          and forced DOM relayouts to periodically touch the
          memory pages that contain secret data. This coerces
          the OS's least-recently-used algorithm to keep the
          sensitive RAM pages in memory.
    \item Veil sites which desire the highest level of privacy can
          opt to use Veil's \emph{DOM hiding} mode. In this
          mode, the client browser essentially acts as a dumb
          graphical terminal. Pages are rendered on a content
          provider's machine, with the browser sending user inputs
          to the machine via the blinding servers; the content
          provider's machine responds with new bitmaps that
          represent the updated view of the page. In DOM hiding
          mode, the page's unique HTML, CSS, and JavaScript content
          is never transmitted to the client browser.
    \item Veil also lets a page store private, persistent
          state by \emph{encrypting} that state and by naming
          it with a blinded reference that only the user can
          generate.
  \end{itemize}
By using blinded references for all content names (including
those of top-level web pages), Veil avoids information leakage
via client-side, name-centric interfaces like the
DNS cache~\cite{timingAttacks}, the browser cache, and
the browser's address bar.
% jwm: We might want to omit the next sentence, since I think
%      that the CSS link color attack has been fixed by browsers
%      for a while now.
% Blinded references also protect against more subtle data
% exfiltrations, like the use of CSS styles to detect which
% resources a browser has previously downloaded~\cite{cssHistoryAttack,cssHistoryAttack2}.
Encryption allows a Veil page to safely leverage the
browser cache to reduce page load times, or store user data
across different sessions of the private web page. A page
that desires the highest level of security will eschew even
the encrypted cache, and use DOM hiding; in concert with URL
blinding, the hiding of DOM content means that the page will
generate \emph{no greppable state in RAM or persistent storage}
that could later be used to identify the page. Table~\ref{t:modeTable}
summarizes the different properties of Veil's two modes for
private browsing.
%!!!jwm: So, we should probably have an experiment in which
%        we load a page in DOM hiding mode using a VM, and
%        then we grep the VM snapshot for content that resides
%        in the page.

% Wowsers, it's much easier to design LaTeX tables using a
% graphical table generator! I used this one for the table
% below:
%        http://www.tablesgenerator.com/#

\begin{table*}[t!]
\footnotesize
\centering
\scalebox{0.7}{
\begin{tabular}{lllll}
\begin{tabular}[c]{@{}l@{}}\ \\ \ \\ Browsing mode\end{tabular}                                                                                 & \begin{tabular}[c]{@{}l@{}}\ \\ Persistent, per-site client-side\\ storage\end{tabular} & \begin{tabular}[c]{@{}l@{}}Information leaks\\ through client-side,\\ name-based  interfaces\end{tabular} & \begin{tabular}[c]{@{}l@{}}\ \\  \ \\ Per-site browser RAM artifacts\end{tabular} & \begin{tabular}[c]{@{}l@{}}\ \\  \ \\ GUI interactions\end{tabular} \\ \hline
\multicolumn{1}{|l|}{Regular browsing}                                                                                                          & \multicolumn{1}{l|}{Yes (cleartext by default)}                                         & \multicolumn{1}{l|}{Yes}                                                                                  & \multicolumn{1}{l|}{Yes}                                                          & \multicolumn{1}{l|}{Locally processed}                              \\ \hline
\multicolumn{1}{|l|}{Regular incognito mode}                                                                                                    & \multicolumn{1}{l|}{No}                                                                 & \multicolumn{1}{l|}{Yes}                                                                                  & \multicolumn{1}{l|}{Yes}                                                          & \multicolumn{1}{l|}{Locally processed}                              \\ \hline
\multicolumn{1}{|l|}{\begin{tabular}[c]{@{}l@{}}Veil with encrypted\\ client-side storage, mutated\\ DOM content, heap walking\end{tabular}} & \multicolumn{1}{l|}{Yes (always encrypted)}                                             & \multicolumn{1}{l|}{No (blinding servers)}                                                                                   & \multicolumn{1}{l|}{Yes (but mutated and heap-walked)}                            & \multicolumn{1}{l|}{Locally processed}                              \\ \hline
\multicolumn{1}{|l|}{Veil with DOM hiding}                                                                                                   & \multicolumn{1}{l|}{No}                                                                 & \multicolumn{1}{l|}{No (blinding servers)}                                                                                   & \multicolumn{1}{l|}{No}                                                           & \multicolumn{1}{l|}{Remotely processed}                             \\ \hline
\end{tabular}
}
\vspace{3mm}
\caption{A comparison between Veil's two browsing modes, regular incognito browsing,
         and regular browsing that does not use incognito mode.}
\label{t:modeTable}
\end{table*}


In summary, \textbf{Veil is the first web framework that allows
developers to implement privacy-preserving browsing
semantics for their own pages.} These semantics are stronger
than those provided by native in-browser incognito modes;
however, Veil pages load on commodity browsers, and do
not require users to reconfigure their systems or run their
browsers within a special virtual machine~\cite{lacuna}. 
Veil can translate legacy pages to more secure versions
automatically, or with minimal developer assistance, easing the barrier to deploying
privacy-preserving sites. Experiments show that
Veil's overheads are moderate: 1.25x--3.25x for Veil
with encrypted client-side storage, mutated DOM content,
and heap walking; and 1.2x--2.1x for Veil in DOM hiding mode.

\subsection{Veil Overview}
\begin{figure*}[t!]
	\centering
	\includegraphics[width=\textwidth]{veil-overview.pdf}
	\caption{The Veil architecture (cryptographic operations omitted for clarity).}
	\label{fig:arch}
\end{figure*}

As shown in Figure~1, the Veil framework  %!!!jwm: Arg, the fig ref isn't resolving correctly.
consists of three components. The \emph{compiler}
transforms a normal web page into a new version
that implements static and dynamic privacy protections.
Web developers upload the compiler's output to
\emph{blinding servers}. These servers act as
intermediaries between content publishers and
content users, mutating and encrypting content. To load the Veil
page, a user first loads a small \emph{bootstrap page}.
The bootstrap asks for a per-user key and the
URL of the Veil page to load; the bootstrap then
downloads the appropriate objects from the blinding
servers and dynamically overwrites itself with
the privacy-preserving content in the target page.

\subsection{Deployment Model}

Veil uses an opt-in model, and is intended
for web sites that want to actively protect
client-side user privacy. For example, a
whistleblowing site like SecureDrop~\cite{secureDrop}
is incentivized to hide client-side evidence
that the SecureDrop website has been visited;
strong private browsing protections give people
confidence that visiting SecureDrop on a work
machine will not lead to incriminating aftereffects.
As another example of a site that is well-suited
for Veil, consider a web page that allows
teenagers to find mental health services. Teenagers
who browse the web on their parents' machines 
will desire strong guarantees that the
machines store no persistent records of private
browsing activity.

Participating Veil sites must explicitly recompile
their content using the Veil compiler. This
requirement is not unduly burdensome, since
all non-trivial frameworks for web development impose
a developer-side workflow discipline. For example,
Aurelia~\cite{aurelia}, CoffeeScript~\cite{coffeeScript}, and
Meteor~\cite{meteor} typically require a
compilation pass before content can go live.

Participating Veil sites must also explicitly serve
their content from Veil blinding servers. Like
Tor servers~\cite{tor}, Veil's blinding servers
can be run by volunteers, although content providers
can also contribute physical machines or VMs to
the blinding pool.

Today, many sites are indifferent
towards the privacy implications of web browsing;
other sites are interested in protecting privacy,
but lack the technical skill to do so; and others
are actively invested in using technology to hide
sensitive user data. Veil targets the latter
two groups of site operators. Those groups are
currently in the minority, but they are growing.
An increasing number of web services define their
value in terms of privacy protections~\cite{duckduck,enigma,privio,privly},
and recent events have increased popular
awareness of privacy issues~\cite{torAfterSnowden}.
Thus, we believe that frameworks like Veil will
become more prevalent as users demand more
privacy, and site operators demand more tools to
build privacy-respecting systems.

\subsection{Threat Model}

Veil assumes that a web service is actively interested
in preserving its users' client-side privacy.
Thus, Veil trusts web developers and the blinding servers.
Veil's goal is to defend the user against local
attackers who take control of a user's machine
\textit{after} a private session terminates. If an
attacker has access to the machine \emph{during} a
private session, the attacker can directly extract
sensitive data, e.g., via keystroke logging or by
causing the browser to core dump; such exploits are
out-of-scope for this paper.

Veil models the post-session attacker as a skilled
system administrator who knows the location and purpose
of the swap file, the browser cache, and files like
\texttt{/var/log/*} that record network activity
like DNS resolution requests. Such an attacker can
use tools like \texttt{grep} or \texttt{find} to
look for hostnames, file types, or page content that
was accessed during a Veil session. The attacker
may also possess off-the-shelf forensics tools like
Mandiant Redline~\cite{mandiant} that look for traces
of private browsing activity. However, the attacker
lacks the skills to perform a customized, low-level
forensics investigation that, e.g., tries to manually
extract C++ data structures from browser memory
pages that Veil could not prevent from swapping out.

Given this attacker model, Veil's security goals are
weaker than strict forensic deniability~\cite{lacuna}.
However, Veil's weaker type of forensic resistance
is both practically useful and, in many cases, the
strongest guarantee that can be provided without
forcing users to run browsers within special OSes or
virtual machines. Veil's goal is to load pages within
\textit{unmodified} browsers that run atop
\textit{unmodified} operating systems. Thus, Veil
is forced to implement privacy-preserving features
using browser and OS interfaces that are unaware of
Veil's privacy goals. These constraints make it
impossible for Veil to provide strict forensic
deniability. However, most post-session attackers
(e.g., friends, or system administrators at work,
Internet cafes, or libraries) will
lack the technical expertise to launch FBI-style
forensic investigations.

Using blinded URLs, Veil tries to prevent data leaks
through system interfaces that use network names.
Examples of name-based interfaces are the browser's ``visited pages''
history, the browser cache, cookies, and the DNS
cache (which leaks the hostnames of the web servers
that a browser contacts~\cite{aggarwal10}). It is
acceptable for the attacker to learn that a user
has contacted Veil's blinding servers---those
servers form a large pool whose hostnames are generic
(e.g., \texttt{veil.io}) and do not reveal
any information about particular Veil sites.

Veil assumes that web developers only include
trusted content that has gone through the Veil
compiler.
% jwm: I think that the sentence below is redundant,
%      given the previous sentence.
% Thus, Veil assumes that a page does not embed
%  unrewritten libraries that update local storage
% or issue unblinded network requests.
A page may embed third party content like a JavaScript
library, but the Veil compiler analyzes both
first party and third party content during compilation.

Heap walking allows Veil
to prevent sensitive RAM artifacts from swapping to
disk. Veil does not try to stop information
leaks from GPU RAM~\cite{lee2014}, but GPU RAM
is never swapped to persistent storage. 
%Veil also cannot prevent the browser from leaking memory
%artifacts through core dumps. We assume that the
%browser does not autonomously try to exfiltrate
%application state, e.g., by sending analytics to
%the browser vendor.
Poorly-written or malicious browser extensions that leak
sensitive page data~\cite{lerner13} are also outside the
scope of this paper.
