\section{Introduction}

Web services like Google, Facebook, and Dropbox are now an essential part of people’s lives. Users
willingly provide their data to these services because these services deliver substantial value in return
through their centralization and analysis of data, such product recommendations and ability to easily
share information. For example, users are willing to share their data with Facebook to learn
about the social lives of their friends as well as share their own social lives more easily.
Similarly, users are willing to provide their data to Amazon to discover
better product recommendations.
To provide value to users, 
these services collect, store, and analyze large 
amounts of their users' sensitive data. However,
once the user provides her information to the web service,
she \textit{loses control} over how the application manipulates
that data. For example, a user cannot control where the application
forwards her data. Even if the service wanted to allow users to 
define access controls, it is unclear how these access controls
should be expressed and enforced. Not only is it difficult to develop
these secure access control mechanisms, but it is also difficult to ensure
these mechanisms are \textit{practical}. 
This thesis addresses these concerns.
More specifically, it focuses on \textit{building practical, secure 
mechanisms for protecting user data in large-scale, distributed web services}.

In this thesis, I will describe three systems that address a variety
of concerns around data leakage in web applications. The first two
systems focus on protecting user data against server-side leakage.  
Splinter leverages a recent cryptographic
primitive, function secret sharing, to practically
execute these queries without revealing sensitive information to the servers.
The next system, Riverbed, 
provides practical information flow control for distributed systems without
requiring developers to label state or write code in special languages.
The final system, Veil, focuses on client-side leakage. 
Veil allows web page developers to enforce stronger private browsing semantics without
browser support.